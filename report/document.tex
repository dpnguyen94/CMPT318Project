%%%%%%%%%%%%%%%%%%%%%%%%%%%%%%%%%%%%%%%%%
% University Assignment Title Page 
% LaTeX Template
% Version 1.0 (27/12/12)
%
% This template has been downloaded from:
% http://www.LaTeXTemplates.com
%
% Original author:
% WikiBooks (http://en.wikibooks.org/wiki/LaTeX/Title_Creation)
%
% License:
% CC BY-NC-SA 3.0 (http://creativecommons.org/licenses/by-nc-sa/3.0/)
% 
% Instructions for using this template:
% This title page is capable of being compiled as is. This is not useful for 
% including it in another document. To do this, you have two options: 
%
% 1) Copy/paste everything between \begin{document} and \end{document} 
% starting at \begin{titlepage} and paste this into another LaTeX file where you 
% want your title page.
% OR
% 2) Remove everything outside the \begin{titlepage} and \end{titlepage} and 
% move this file to the same directory as the LaTeX file you wish to add it to. 
% Then add \input{./title_page_1.tex} to your LaTeX file where you want your
% title page.
%
%%%%%%%%%%%%%%%%%%%%%%%%%%%%%%%%%%%%%%%%%
%\title{Title page with logo}
%----------------------------------------------------------------------------------------
%	PACKAGES AND OTHER DOCUMENT CONFIGURATIONS
%----------------------------------------------------------------------------------------

\documentclass[12pt]{article}
\usepackage[english]{babel}
\usepackage[colorinlistoftodos]{todonotes}
\usepackage[utf8]{inputenc}
\usepackage[english]{babel}
\usepackage{fancyhdr}
\usepackage[margin=1in,headheight=15pt]{geometry}
\usepackage{microtype}
\usepackage{color,soul}
\usepackage{enumitem}
\usepackage{amsmath,amsthm,amssymb}
\usepackage{graphicx}
\graphicspath{ {images/} }
\usepackage{float}
\newtheorem{defn}{Definition}[section]

\usepackage{csquotes}
\usepackage{minted}

\begin{document}

\begin{titlepage}

\newcommand{\HRule}{\rule{\linewidth}{0.5mm}} % Defines a new command for the horizontal lines, change thickness here

\center % Center everything on the page
 
%----------------------------------------------------------------------------------------
%	HEADING SECTIONS
%----------------------------------------------------------------------------------------

\textsc{\LARGE Simon Fraser University}\\[1.5cm] % Name of your university/college
\textsc{\Large CMPT 318 Data Science}\\[0.5cm] % Major heading such as course name
%\textsc{\large Minor Heading}\\[0.5cm] % Minor heading such as course title

%----------------------------------------------------------------------------------------
%	TITLE SECTION
%----------------------------------------------------------------------------------------

\HRule \\[0.4cm]
{ \huge \bfseries Project Report}\\[0.4cm] % Title of your document
\HRule \\[1.5cm]
 
%----------------------------------------------------------------------------------------
%	AUTHOR SECTION
%----------------------------------------------------------------------------------------

\begin{minipage}{0.4\textwidth}
\begin{flushleft} \large
Jagrajan \textsc{Bhullar}
\end{flushleft}
\end{minipage}
~
\begin{minipage}{0.4\textwidth}
\begin{flushright} \large
Duc Phuong \textsc{Nguyen}
\end{flushright}
\end{minipage}\\[2cm]

% If you don't want a supervisor, uncomment the two lines below and remove the section above
%\Large \emph{Author:}\\
%John \textsc{Smith}\\[3cm] % Your name

%----------------------------------------------------------------------------------------
%	DATE SECTION
%----------------------------------------------------------------------------------------

{\large \today}\\[2cm] % Date, change the \today to a set date if you want to be precise

%----------------------------------------------------------------------------------------
%	LOGO SECTION
%----------------------------------------------------------------------------------------

%\includegraphics{logo.png}\\[1cm] % Include a department/university logo - this will require the graphicx package
 
%----------------------------------------------------------------------------------------

\vfill % Fill the rest of the page with whitespace

\end{titlepage}

\section{The Problem}

\hl{Talk about the Problem}

\section{The Data}

As mentioned in the Project Details, the weather data we have (from YVR airport station) does not perfectly match the weather images (from English Bay) so some clean-up jobs were required before training models. 

First, weather data (observation) was read as a DataFrame which contains two main columns (Date/Time and Weather). Rows with empty observations will be excluded and Date/Time string will be converted to \textit{datetime64} object (using \textit{pd.to\_datetime()}) for merging later on.

Next, the same process was implemented with images data from KatKam. The original images scale was quite big (256 x 192 = 49192 inputs) and more than necessary for training models so it was re-sized to only half (192 x 96 = 12288 inputs) while not losing any accuracy at later steps (tested). 

Each pixel was then represented as a \textit{float64} value (instead of 3 distinct integer values for \textit{RGB colors}, to limit the number of input features). Finally, they were reshaped to fit in the DataFrame with 'datetime' and all the features as columns and merged with the weather data to form a new DataFrame to be used for learning.

\section{Trying Out Classifiers}

Our first intuition would lead us to a classification problem where inputs are the images data and outputs are the weather description, so let us start with some simple methods to deal with this and to produce some kinds of result. 

\subsection{Single-Label Classifier}

My first attempt to solve this problem is to train a model using single-label classifier (where different set of labels will give different outputs; for example, label set [Rain, Cloudy] would be distinguished from [Rain] or [Cloudy], which is not very practical, but worth trying).

Three approaches were used to build the model pipeline for predicting outputs: \textit{KNeighborsClassifier} (with n\_neighbors = 5, 7, 9, 11), \textit{DecisionTreeClassifier()} and \textit{RandomForestClassifier} (with n\_estimator = 10, 20, 30). Although the results were not very surprising, they all gave different accuracy scores as followed: (approximately) \textbf{0.43} for \textit{kNN}, \textbf{0.25} for \textit{DTree} and \textbf{0.18} for \textit{RForest}. In addition, \textit{PCA(250)} was used to decrease the number of dimensions in the inputs so runtime can be varied from 20 to 30 seconds, depending on parameters used. 

\subsection{Multi-Label Classifier}

However, this is actually considered a multi-label classification problem where the output can (and usually) belong to more than one categories (for example: [Drizzle, Fog] must be assigned two distinct labels Drizzle and Fog, instead of just one label). This led us to training a model using \textit{OneVsRestClassifier()} with built-in estimator (such as \textit{kNN}, \textit{DTree} and \textit{RForest} as mentioned above) to improve the accuracy score. 

Since the original weather descriptions were in string format, we need to perform some string processing to get desired outputs: first, separate them commas (,) and then, put them in list. These lists were then converted into a 2-D binary indicator array where element in row i, column j indicate the presence of label j-th in sample i-th, using \textit{MultiLabelBinarizer()}.

The rest of the jobs is to run it through the model and get the results as below (estimated):

\begin{table} [h!]

Number of categories: 17 \\

\begin{tabular}{c|c} 
    \textit{KNeighborsClassifier(n\_neighbors=9)} & \textbf{0.33} \\
    \textit{DecisionTreeClassifier()} & \textbf{0.16} \\
    \textit{RandomForestClassfier(n\_estimators=20)} & \textbf{0.12}
\end{tabular}

\end{table}

\subsection{Multi-Label Classifier with reduced categories}
The main issue here might lie in the number of categories we have, in relevance to the weather descriptions. It could be a major obstacle for the classifiers to do a good job. Moreover, the images data obtained is not necessarily perfectly aligned with the weather observations (since they are recorded at different location and time) and it is very hard for the camera to capture all the information, merely by taking pictures of the sky (visually, there is little to none change in the images between types of Rain or Snow, or even between Rain/Snow and Cloudy).

Thus, it could be a good idea to group up similar weather conditions and form a new set of labels to help the classifier identifying the outputs more robustly. Below are list of weather description before and after using filter:

\begin{table} [h!]
\centering
\caption{List of weather conditions}
\begin{tabular}{l|l} 

\textbf{Original}     &                 \textbf{Reduced}         \\ \hline
Clear                 &                 Clear                    \\
Mainly Clear          &                                          \\ \hline
Cloudy                &                 Cloudy                   \\
Mostly Cloudy         &                                          \\ \hline
Fog                   &                 Fog                      \\ 
Freezing Fog          &                                          \\ \hline
Drizzle               &                 Drizzle                  \\ \hline
Rain                  &                                          \\
Heavy Rain            &                                          \\
Moderate Rain         &                 Rain                     \\ 
Moderate Rain Showers &                                          \\
Rain Showers          &                                          \\ \hline
Snow                  &                                          \\
Snow Pellets          &                 Snow                     \\
Snow Showers          &                                          \\ 
Moderate Snow         &                                          \\ \hline
Thunderstorms         & Thunderstorms                            \\
\end{tabular}
\end{table}

The final results acquired are not too terrible, indeed! Using the same methods we have discussed last section, we can reach up to \textbf{0.62} accuracy score, equivalent to a \textbf{90\%} boost while reducing number of categories to only half.

\begin{table} [h!]

Number of categories: 17 \\

\begin{tabular}{c|c} 
    \textit{KNeighborsClassifier(n\_neighbors=9)} & \textbf{0.62} \\
    \textit{DecisionTreeClassifier()} & \textbf{0.35} \\
    \textit{RandomForestClassfier(n\_estimators=20)} & \textbf{0.33}
\end{tabular}

\end{table}

Of course, this does not compensate for the lack of details in the outputs since it is a trade-off and we probably needs to tackle the problem in more efficient ways.

\end{document}